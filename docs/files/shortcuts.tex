% ----------------------------------------
% custom math commands
% ----------------------------------------

% - - - - - - - - - - - - - - - - - - - - 
% not requiring different packages
% - - - - - - - - - - - - - - - - - - - - 
\newcommand{\E}[1]{\mathbb{E} \left[ #1 \right]}
\newcommand{\PP}[1]{\mathbb{P} \left[ #1 \right]}
\newcommand{\var}[1]{\mathbb{V}\text{ar} \left[ #1 \right]}
\newcommand{\cov}[1]{\mathbb{C}{ov} \left[ #1 \right]}
\newcommand{\corr}[1]{\mathbb{C}{orr} \left[ #1 \right]}
\newcommand{\spc}[0]{\text{ }}
\newcommand{\R}[0]{\mathbb{R}}
\newcommand{\C}[0]{\mathbb{C}}
\newcommand{\Sym}[0]{\mathbb{S}}
\newcommand{\VaR}[2]{\operatorname{VaR}_{#1} \left[ #2 \right]}
\newcommand{\CVaR}[2]{\operatorname{CVaR}_{#1} \left[ #2 \right]}

\newcommand{\sumi}{\sum_{i=1}}
\newcommand{\sumj}{\sum_{j=1}}
\newcommand{\sumk}{\sum_{k=1}}
\newcommand{\prodi}{\prod_{i=1}}
\newcommand{\prodj}{\prod_{j=1}}
\newcommand{\prodk}{\prod_{k=1}}

\newcommand{\etal}{et al. }
\newcommand{\ie}{i.e., }
\newcommand{\eg}{e.g., }

\newcommand{\im}{\text{i}\,}

\newcommand{\sgn}{\text{sgn}}

% \newcommand{\note}[1]{\textcolor{red}{note: #1}}

% inline equations
\makeatletter
\newcommand*{\inlineequation}[2][]{%
	\begingroup
	% Put \refstepcounter at the beginning, because
	% package `hyperref' sets the anchor here.
	\refstepcounter{equation}%
	\ifx\\#1\\%
	\else
	\label{#1}%
	\fi
	% prevent line breaks inside equation
	\relpenalty=10000 %
	\binoppenalty=10000 %
	\ensuremath{%
		% \displaystyle % larger fractions, ...
		#2%
	}%
	~\@eqnnum
	\endgroup
}
\makeatother

% - - - - - - - - - - - - - - - - - - - - 
% requiring different packages
% - - - - - - - - - - - - - - - - - - - - 
\usepackage{dsfont}
\newcommand{\ind}[1]{\mathds{1}{\{#1\}}}
\newcommand{\one}{\mathds{1}}
% for indicator function

\usepackage{mathtools}
% for \coloneqq :=

% \usepackage{algorithm}
% \usepackage[noend]{algpseudocode}
% \makeatletter
% \def\BState{\State\hskip-\ALG@thistlm}
% \makeatother
% % for algorithm pseudocode b-states

